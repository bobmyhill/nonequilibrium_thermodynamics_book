\chapter{Introduction}

\section{Historical Background of Non-Equilibrium Thermodynamics}

Thermodynamic considerations were first applied to the treatment of irreversible processes by W. Thomson\footnote{W. Thomson (Lord Kelvin) Proc. roy. Soc. Edinburgh 3 (1854) 225; Ibid. Trans. 21 I (1857) 123; Math. phys. Papers 1 (1882) 232.} in 1854. He analysed the various thermo-electric phenomena and established the famous two relations which bear his name. The first of these relations follows from conservation of energy. The second relation, which relates the thermoelectric potential of a thermocouple to its Peltier heat, was obtained from the two laws of thermodynamics, and an additional assumption about the so-called ``reversible'' contributions to the process. Later Boltzmann\footnote{L. Boltzmann, Sitz. her. Akad. Wiss. Wien, Math.-Naturw. Kl. Abt. II 96 (1887) 1258; Wiss. Abh. 3 (1909) 321.} attempted without success to justify the Thomson hypothesis. We now know that no basis exists for this hypothesis. Thomson's second relation was finally derived correctly by Onsager who showed that this relation was a consequence of the invariance of the microscopic equations of motion under time reversal. Thomson's method was applied with varying success to a number of other irreversible phenomena, but a coherent scheme for the macroscopic description of irreversible processes could not be developed in this way.

Independently of the theoretical development described above, a number of physicists undertook, at the turn of the century, to give more refined formulations of the second law of thermodynamics for non-equilibrium situations. As early as 1850, Clausius introduced the concept of ``non-compensated heat'' as a measure of irreversibility (in systems which need not be thermally insulated from their surroundings). Duhem, Natanson, Jaumann and Lohr and later Eckartt\footnote{P. Duhem, Energetique (2 vol.) (Gauthier-Villars, Paris, 1911).;  L. Natanson, Z. phys. Chem. 21 (1896) 193.; G. Jaumann, Sitz. her. Akad. Wiss. Wien, Math.-Naturw. Kl. Abt. II A 120 (1911) 385; Denkschr. Akad. Wiss. Wien, Math.-Naturw. Kl. 95 (1918) 461.; E. Lohr, Denkschr. Akad. Wiss. Wien, Math.-Naturw. Kl. 93 (1916) 339; 99 (1924) 11, 59; Festschr. Techn. Hochsch. Brunn (1924) 176.; C. Eckart, Phys. Rev. 58 (1940) 267, 269, 919, 924.} attempted to obtain expressions for the rate of change of the local entropy in nonuniform systems by combining the second law of thermodynamics
with the macroscopic laws of conservation of mass, momentum and energy. In this way they derived formulae which related irreversibility to the non-uniformity of the system. Similarly De Donder\footnote{Th. De'Donder, L'affinite (Gauthier-Villars, Paris, 1927).} was able to relate the ``non-compensated heat'' in a chemical reaction to the affinity, a thermodynamic variable characterizing the state of the system. The systematic discussion of irreversible processes along these
lines however was not completed until much later.

In the meantime, in 1931, Onsager\footnote{L. Onsager, Phys. Rev. 37 (1931) 405; 38 (1931) 2265.} established his celebrated ``reciprocal relations'' connecting the coefficients, which occur in the linear phenomenological laws that describe the irreversible processes. These reciprocal relations, of which Thomson's second relation is an example, reflect on the macroscopic level the time reversal invariance of the microscopic equations of motion. In 1945 Casimir\footnote{H. B. G. Casimir, Rev. mod. Phys. 17 (1945) 343; or Philips Res. Rep. 1 (1945) 185.} reformulated the reciprocal relations, so that they would be valid for a larger class of irreversible phenomena than had been previously considered by Onsager.

Finally Meixner\footnote{J. Meixneir, Ann. Physik [5] 39 (1941) 333; 41 (1942) 409; 43 (1943) 244; Z. phys. Chem. B S3 (1943) 235.} in 1941 and the following years, and somewhat later Prigogine\footnote{I. Prigogine, Etude thennodynamique des phenomenes irreversibles (Dunod, Paris and Desoer, Liege, 1947).}, set up a consistent phenomenological theory of irreversible processes, incorporating both Onsager's reciprocity theorem and the explicit calculation for a certain number of physical situations of the so-called entropy source strength (which is in fact the noncompensated heat of Clausius). In this way a new field of ``thermodynamics of irreversible processes'' was born. It developed rapidly in various directions.

Coupled with the recent growing interest in the statistical mechanical theory of irreversible processes are some significant studies on the statistical basis of the thermodynamics of irreversible processes. Thus special attention is paid to the validity on one hand of thermodynamic relations outside equilibrium and, on the other hand, to the Onsager reciprocal relations. Many of these studies have employed methods and concepts borrowed from the theory of stochastic processes.

\section{Systematic Development of the Theory}

The field of non-equilibrium thermodynamics provides us with a general framework for the macroscopic description of irreversible processes. As such it is a branch of macroscopic physics, which has connections with other macroscopic disciplines such as fluid dynamics and electromagnetic theory, insofar as the latter fields are also concerned with non-equilibrium situations. Thus the thermodynamics of irreversible processes should be set up from the start as a continuum theory, treating the state parameters of the theory as field variables, i.e., as continuous functions of space coordinates and time. Moreover one would like to formulate the basic equations of the theory in such a way that they only contain quantities referring to a single point in space at one time, i.e. in the form of local equations. This is also the way in which fluid dynamics and the Maxwell theory are formulated. In equilibrium thermodynamics such a local formulation is generally not needed, since the state variables are usually independent of the space coordinates.

In non-equilibrium thermodynamics the so-called balance equation for the entropy plays a central role. This equation expresses the fact that the entropy of a volume element changes with time for two reasons. First it changes because entropy flows into the volume element, second because there is an entropy source due to irreversible phenomena inside the volume element. The entropy source is always a non-negative quantity, since entropy can only be created, never destroyed. For reversible transformations the entropy source vanishes. This is the local formulation of the second law of thermodynamics. The main aim is to relate the entropy source explicitly to the various irreversible processes that occur in a system. To this end one needs the macroscopic conservation laws of mass, momentum and energy, in local, i.e. differential form. These conservation laws contain a number of quantities such as the diffusion flows, the heat flow and the pressure tensor, which are related to the transport of mass, energy and momentum. The entropy source may then be calculated if one makes use of the thermodynamic Gibbs relation which connects. in an isotropic multi-component fluid for instance, the rate of change of entropy in each mass element, to the rate of change of energy and the rates of change in composition. It turns out that the entropy source has a very simple appearance: it is a sum of terms each being a product of a flux characterizing an irreversible process, and a quantity, called thermodynamic force, which is related to the non-uniformity of the system (the gradient of the temperature for instance) or to the deviations of some internal state variables from their equilibrium values (the chemical affinity for instance). The entropy source strength can thus serve as a basis for the systematic description of the irreversible processes occurring in a system.

As yet the set of conservation together with the entropy balance equation and the equations of state is to a certain extent empty, since this set of equations contains the irreversible fluxes as unknown parameters and can therefore not be solved with given initial and boundary conditions for the state of the system. At this point we must therefore supplement our equations by an additional set of phenomenological equations which relate the irreversible fluxes and the thermodynamic forces appearing in the entropy source strength. In first approximation the fluxes are linear functions of the thermodynamic forces. Fick's law of diffusion, Fourier's law of heat conduction, and Ohm's law of electric conduction, for instance, belong to this class of linear phenomenological laws. It also contains in addition possible cross-effects between various phenomena, since each flux may in principle be a linear function of all the thermodynamic forces which are needed to characterize the entropy source strength. The Soret effect, which results from diffusion in a temperature gradient is an example ,of such a cross-effect. Many others exist such as the thermoelectric effects, the group of thermomagnetic and galvanomagnetic effects, and also the electrokinetic effects. Non-equilibrium thermodynamics, in its present form, is mainly restricted to the study of such linear phenomena. Very little of a sufficiently general nature is known outside this linear domain. This is not a very serious restriction however, since even in rather extreme physical situations, transport processes, for example, are still described by linear laws. Together with the phenomenological equations the original set of conservation laws may be said to be complete in the sense that one now has at one's disposal a consistent set of partial differential equations for the state parameters of a material system, which may be solved with the proper initial and boundary conditions.

Some rather important statements of a macroscopic nature may be made concerning the matrix of phenomenological coefficients which relate the fluxes and the thermodynamic forces, appearing in the entropy source strength. In the first place the Onsager-Casimir reciprocity theorem gives rise to a number of relations amongst these coefficients, thus reducing the number of independent quantities and relating distinct physical effects to each other. It is one of the aims of nonequilibrium thermodynamics to study the physical consequences of the reciprocal relations in applications of the theory to various physical situations. Apart from the reciprocity theorem, which is based on the time reversal invariance of the microscopic equations of motion, possible spatial symmetries of a material system may further simplify the scheme of phenomenological coefficients. Thus in an isotropic fluid a scalar phenomenon like a chemical reaction cannot be coupled to a vectorial phenomenon like heat conduction. This reduction of the scheme of phenomenological coefficients, which results from invariance of the phenomenological equations under special orthogonal transformations, goes under the name of the Curie principle, but should more appropriately be called Curie's theorem.

The reciprocal relations have transformation properties which have been studied extensively. Thus Meixner showed that the Onsager relations are invariant under certain transformations of the fluxes and the thermodynamic forces. There exist a number of other general theorems, which are of use in non-equilibrium thermodynamics: one can show that at mechanical equilibrium the entropy production has some additional invariance properties. It can also be shown that stationary non-equilibrium states are characterized by a minimum property: under certain restrictive conditions the entropy production has, in the stationary state, a minimum value compatible with given boundary conditions. Both of these theorems have first been obtained by Prigogine.


Just as the principles of equilibrium thermodynamics must be justified by means of statistical mechanical methods, so the principles of thermodynamics of irreversible processes require a discussion of their microscopic basis. In the present state of theoretical development a microscopic discussion of irreversibility itself from first principles would lie outside the scope of this treatise. However, even if the irreversible behaviour of macroscopic systems is taken for granted, one still has the problem of discussing the remaining foundations of the theory. On these premises Onsager's reciprocity theorem can indeed be derived from microscopic properties of a mechanical many-particle system. Concepts of fluctuation theory and the theory of stochastic processes play an essential role in such a discussion of the foundations, to which Onsager and Machlup have contributed by using a Brownian motion type model for the regression of fluctuations. Such a model can also serve for a justification of the use of thermodynamic relations outside equilibrium. Furthermore the methods of the theory of stochastic processes are used in relating the spontaneous fluctuations in equilibrium to the macroscopic response of a system under external driving forces. The relation thus obtained is known as the fluctuation-dissipation theorem and is due to Callen, Greene and Welton. It represents in fact a generalization of the famous Nyquist formula in the theory of electric noise.

The problem of justifying the principles of non-equilibrium thermodynamics can alternatively be approached from the viewpoint of the kinetic theory of gases. Such a method is more limited since it only applies to gaseous systems at low density, however it permits one to express those macroscopic quantities which pertain to irreversible processes in terms of molecular parameters. The irreversibility itself is already contained in the fundamental equation of the kinetic theory of gases, the Boltzmann integro-differential equation. One may then justify the use of thermodynamic relations for gases outside equilibrium (as was first done by Prigogine), and derive the Onsager reciprocal relations.

The theory of non-equilibrium thermodynamics has found a great variety of applications in physics and chemistry. For a systematic classification of these applications one may group the various irreversible phenomena according to their ``tensorial character''. First one has ``scalar phenomena''. These include chemical reactions and structural relaxation phenomena. Onsager relations are of help in this case in solving the set of ordinary differential equations which describe the simultaneous relaxation of a great number of internal variables.

A second group of phenomena is formed by ``vectorial processes'' such as diffusion, heat conduction, and their cross-effects. Viscous phenomena can be considered as a third group to which methods of non-equilibrium thermodynamics have been applied. In particular the theory of acoustical relaxation has been consistently developed within this framework by Meixner.

Altogether new aspects arise when an electromagnetic field acts on a material system. Then the continuity laws for electromagnetic energy and momentum which follow from the Maxwell equations must also be taken into account. One must therefore reformulate the theory to suit the need of this case with its numerous applications to both polarized and unpolarized media.

